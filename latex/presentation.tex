\documentclass{beamer}
\usepackage{wrapfig}
\usepackage{commath}
\usepackage{braket}
%\usepackage{physics}
\usepackage{amsmath,amssymb,amsfonts,scalerel,stackengine}
\usepackage{mathpazo}
\usepackage{subcaption}
\usepackage{multicol}
\usepackage{mathtools}
\usepackage[makeroom]{cancel}
\usepackage{lipsum}
\usepackage{tikz}
\usetikzlibrary{quantikz}
\usepackage{bm}


%\usepackage{sectsty}
\usepackage[export]{adjustbox}
\usetheme{Boadilla}
\usefonttheme[onlymath]{serif} 
\setbeamertemplate{navigation symbols}{}
\usepackage{graphicx}
% NOTE: use biber and cite with \footfullcite{key}
\usepackage[sorting=none]{biblatex}
\usepackage{amsmath}
\usepackage{ulem}
%\addbibresource{project3_references.bib}
\setbeamertemplate{footline}
{
  \leavevmode%
  \hbox{%
  \begin{beamercolorbox}[wd=.2\paperwidth,ht=2.25ex,dp=1ex,center]{author in head/foot}%
    \usebeamerfont{author in head/foot} \insertshortdate{}%
  \end{beamercolorbox}%
  \begin{beamercolorbox}[wd=.6\paperwidth,ht=2.25ex,dp=1ex,center]{title in head/foot}%
    \usebeamerfont{title in head/foot}\insertshorttitle
  \end{beamercolorbox}%
  \begin{beamercolorbox}[wd=.2\paperwidth,ht=2.25ex,dp=1ex,center]{date in head/foot}%
 
    \insertframenumber{} / \inserttotalframenumber\hspace*{1ex}
  \end{beamercolorbox}}%
  \vskip0pt%
}
\DeclareMathOperator*{\argmin}{argmin}
\renewcommand*{\bibfont}{\normalfont\tiny}
\renewcommand{\footnotesize}{\tiny}
\title{Semester Project: Variational Monte-Carlo for strongly correlated bosons in continuous space}
\author{Giorgio Facelli}
\date{28/06/2023}

\setbeamercolor{normal text}{fg=black}
\begin{document}

\begin{frame}[plain]
    \titlepage
\end{frame}

\begin{frame}{Contents}
  \begin{enumerate}
    \item \large Article
    \begin{enumerate}[I]
      \item Motivation
      \item Implementation a simulation
      \item Recent technological breakthroughs
    \end{enumerate} 
    \item \large Two- and three-spin Heisenberg model
    \begin{enumerate}[I]
      \item Physical Observables
      \item Error analysis
    \end{enumerate}
    \item \large Conclusions
  \end{enumerate}
    %\begin{block}
    %    {Dummy Block}
    %    \lipsum[1]
    %\end{block}
\end{frame}

\begin{frame}{Introduction}

\end{frame}

\begin{frame}{Variational Monte-Carlo}
Exploits the \textit{variational principle}, which states that the ground-state $\ket{\Psi_0}$ minimizes 
the energy:
\begin{equation}
  \ket{\Psi_0} = \argmin_{\ket{\Psi}}\biggr[\dfrac{\braket{\Psi|\hat{H}|
  \Psi}}{\braket{\Psi|\Psi}}\biggr]
\end{equation}
Given an ansatz $\ket{\Psi(\bm{\theta})}$, the parameters are optimized so as to reach the minimum in 
energy.\\
\textbf{Observables:} Almost every observable can be estimated as a stochastic expectation value over its 
\textit{local} counterpart:
\begin{equation}
\braket{\hat{O}} = E_{\Pi(\bm{x})}\big[O_{\text{loc}}(\bm{x})\big]\text{   where   }
O_\text{loc}(\bm{x}) = \int \,d\bm{x}' O_{\bm{xx}'} \dfrac{\Psi(\bm{x}')}{\Psi(\bm{x})}
\end{equation}
\end{frame}
  
\begin{frame}{Simulation of Heisenberg model}
\begin{columns}

\begin{column}{0.5\textwidth}
{\scriptsize
\begin{align*}
H_{2-spin} = -J(\sigma_x^{(1)}\sigma_x^{(2)}+\sigma_y^{(1)}\sigma_y^{(2)}+\sigma_z^{(1)}\sigma_z^{(2)})
\end{align*}
}
\end{column}

\begin{column}{0.5\textwidth}
{\scriptsize
\begin{align*}
H_{3-spin} &= -J(\sigma_x^{(1)}\sigma_x^{(2)}+\sigma_x^{(2)}\sigma_x^{(3)}+\sigma_y^{(1)}\sigma_y^{(2)}+ \\
&+\sigma_y^{(2)}\sigma_y^{(3)}+\sigma_z^{(1)}\sigma_z^{(2)}+\sigma_z^{(2)}\sigma_z^{(3)})
\end{align*}
}
\end{column}

\end{columns}
\end{frame}

\begin{frame}{Conclusions}
\begin{itemize}
\item A quantum computer would potentially allow to study many unexplored systems.
\item NISQ still far away from having fully functional UQS at disposal.
\item Accuracy of a simulation to be assessed by considering digital and hardware error.
\end{itemize}
\vspace{20mm}
%\printbibliography

\end{frame}
\end{document}

