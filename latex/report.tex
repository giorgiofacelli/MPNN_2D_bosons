\documentclass[a4paper, 12pt, oneside]{article}
%On peut changer "oneside" en "twoside" si on sait que le résultat sera recto-verso.
%Cela influence les marges (pas ici car elles sont identiques à droite et à gauche)

% pour l'inclusion de figures en eps,pdf,jpg,....
%\usepackage{graphicx}

\usepackage{float}
\usepackage{braket}

%Marges. Désactiver pour utiliser les valeurs LaTeX par défaut
%\usepackage[top=2.5cm, bottom=2cm, left=2cm, right=2cm, showframe]{geometry}
\usepackage[top=2.5cm, bottom=2cm, left=2cm, right=2cm]{geometry}

% quelques symboles mathematiques en plus
\usepackage{amsmath}
\usepackage{systeme}
\usepackage{multicol}
\usepackage{enumitem}
\usepackage{amssymb}

\usepackage{wrapfig}

\ProvidesPackage{gensymb}
[2022/10/13 v1.0.1 (KJH)]

\usepackage[font=small,labelfont=bf]{caption}
\usepackage{subcaption,graphicx}


%\usepackage[bookmarks=false,linkcolor=black,urlcolor=black]{hyperref}

% le tout en langue francaise
%\usepackage[francais]{babel}

% on peut ecrire directement les charactères avec l'accent
\usepackage[T1]{fontenc}


\usepackage{xfrac}
% a utiliser sur Linux/Windows
%\usepackage[latin1]{inputenc}

% a utiliser avec UTF8
\usepackage[utf8]{inputenc}
%Très utiles pour les groupes mixtes mac/PC. Un fichier texte enregistré sous codage UTF-8 est lisible dans les deux environnement.
%Plus de problème de caractères accentués et spéciaux qui ne s'affichent pas

% a utiliser sur le Mac
%\usepackage[applemac]{inputenc}

% pour l'inclusion de liens dans le document (pdflatex)
\usepackage[colorlinks,bookmarks=false,linkcolor=black,urlcolor=blue, citecolor=black]{hyperref}

%Pour l'utilisation plus simple des unités et fractions
\usepackage{units}
\usepackage[style=phys]{biblatex}%dinat
\addbibresource{project_report_references.bib}
%Pour utiliser du time new roman... Comenter pour utiliser du ComputerModern
%\usepackage{mathptmx}

%Pour du code non interprété
\usepackage{verbatim}
\usepackage{verbdef}% http://ctan.org/pkg/verbdef


%Pour changer la taille des titres de section et subsection. Ajoutez manuellement les autres styles si besoin.
\makeatletter
\renewcommand{\section}{\@startsection {section}{1}{\z@}%
             {-3.5ex \@plus -1ex \@minus -.2ex}%
             {2.3ex \@plus.2ex}%
             {\normalfont\normalsize\bfseries}}
\makeatother

\makeatletter
\renewcommand{\subsection}{\@startsection {subsection}{1}{\z@}%
             {-3.5ex \@plus -1ex \@minus -.2ex}%
             {2.3ex \@plus.2ex}%
             {\normalfont\normalsize\bfseries}}
\makeatother

% nouvelles commandes LaTeX, utilis\'ees comme abreviations utiles

\newcommand{\mail}[1]{{\href{mailto:#1}{#1}}}
\newcommand{\ftplink}[1]{{\href{ftp://#1}{#1}}}
%
% latex SqueletteRapport.tex      % compile la source LaTeX
% xdvi SqueletteRapport.dvi &     % visualise le resultat
% dvips -t a4 -o SqueletteRapport.ps SqueletteRapport % produit un PostScript
% ps2pdf SqueletteRapport.ps      % convertit en pdf

% pdflatex SqueletteRapport.pdf    % compile et produit un pdf

% ======= Le document commence ici ======
\newcommand{\inlineeqnum}{\refstepcounter{equation}~~\mbox{(\theequation)}}
\begin{document}

% Le titre, l'auteur et la date
\title{Semester Project: Variational Monte-Carlo for Continuous Space Many-Body Quantum Physics}
\author{Giorgio Facelli\\ 
{\small \mail{giorgio.facelli@epfl.ch}}}

\maketitle % Table des matieres
\begin{center}
Hosting group: CQSL lead by Prof. Giuseppe Carleo \\
Supervision: Gabriel Pescia
\end{center}
% Quelques options pour les espacements entre lignes, l'indentation 
% des nouveaux paragraphes, et l'espacement entre paragraphes
\baselineskip=16pt
\parindent=0pt
\parskip=12pt

\section{Introduction}
Multiple areas of physics are nowadays faced with the difficult task of simulating systems obeying the laws of 
quantum mechanics, the theory that to this day most accurately describes the nature around us. Academic reasearch 
is in fact exploiting, now more than ever, exotic quantum effects which arise in particular systems. This is why 
fields such as condensed matter physics, quantum chemistry and quantum optics are in desperate need to explore 
and analyse a wide variety of systems that could trigger breakthroughs in scientifical developments. 
However, simulating such systems is a demanding computational task. In general terms, a system of $N$ interacting 
particles is described by a statevector living in a Hilbert space which scales up exponentially with the number of 
its constituents. Hence, the complete description of the wavefunction needs exponentially many coefficients, which 
is normally computationally unfeasible. Nonetheless, in the recent years, new advanced computational techniques 
along with Machine Learning (ML) methods based on deep learning, have been shown to tackle such systems in a more efficient way. 
This report sets out to present a statistical-based computational approach to simulating quantum systems, 
namely \textit{Variational Monte-Carlo} (VMC) approach \cite{mcmillan_ground_1965}, and how we can combine this 
with a neural network architecture.\\
VMC is a computational method primarily used to estimate the ground state properties of a quantum system. 
In particular, the ground-state energy and the corresponding ground-state wavefunction of a system are computed 
by parametrizing the wavefunction of said system and reducing its energy successively by means of 
\textit{stochastic gradient descent} (SGD) \cite{ruder_overview_2016}. Early applications approximated the ground-state of 
the homogenous electron gas \cite{ceperley_ground_1978, ortiz_correlation_1994}. Later on, the cohesive energies of multiple compounds 
were estimated \cite{li_cohesive_1991, rajagopal_variational_1995, malatesta_variational_1997}.  \\
First efforts of combining VMC with ML culminated in the developement of the so-called Neural Quantum 
States (NQS). These use neural networks to parameterize the quantum mechanical wave-function. They 
have been shown to provide state-of-the-art results on quantum spin-systems such as the one-dimensional 
transverse field Ising model and antiferromagnetic Heisenberg model \cite{carleo_solving_2017}. Such models are motivated by the 
capability of neural networks to approximate an arbitrary function. 
The accuracy of such methods has also been shown for systems in continuous space by analysing a variety of 
atoms and small molecules \cite{hermann_deep-neural-network_2020, pfau_ab_2020, Pescia_2022}. \\
The work presented here will cover the theoretical backgound which lies behind the idea of VMC for many-body quantum 
systems in continuous space. Artificial neural networks will also be introduced. We will then move on to consider a 
system made of $N$ non-interacting bosonic quantum oscillators. Results obtained with the VMC method using simple 
gaussian and multivariate gaussian ansatze will be presented, and their accuracy analysed. Lastly, we will also show 
the same results using a NQS-based ansatz. To test the correct functioning of the VMC developed for this project, 
the NQS model will also be compared to the same ansatz trained using NetKet, an open-source Python library in which 
every aspect of the work included in this report is already implemented in a more comprehensive way \cite{Vicentini_2022}.



\section{Theoretical background}\label{sec:theory}

\subsection{The variational principle}

\subsection{VMC approach and observables' approximation}\label{sec:obs_approx}

\subsection{Markov Chain Monte Carlo (MCMC)}

\subsubsection{Metropolis-Hastings algorithm}

\subsection{Optimization scheme}\label{sec:optimization_scheme}

\subsubsection{Zero-variance property}\label{sec:zero_variance}

\subsection{Auto-correlation time}
\subsection{Artificial Neural Networks}


\subsection{A few words on the code}


\subsection{Non-interacting quantum harmonic oscillators}\label{sec:ansatze}

\section{Main results}\label{sec:results}


\subsection{Gaussian ansatz}\label{sec:one_parameter}


\subsection{Multivariate gaussian ansatz}\label{sec:matrix_parameter}


\subsection{Neural-Network ansatz}\label{sec:neural_network_ansatz}


\section{Discussion}

\section{Conclusions}

\printbibliography

\end{document}